\documentclass[sigconf,anonymous,review]{acmart}
%%\documentclass[sigconf]{acmart}

%% Rights management information.  This information is sent to you
%% when you complete the rights form.  These commands have SAMPLE
%% values in them; it is your responsibility as an author to replace
%% the commands and values with those provided to you when you
%% complete the rights form.
\setcopyright{acmcopyright}
\copyrightyear{2018}
\acmYear{2018}
\acmDOI{XXXXXXX.XXXXXXX}

%% These commands are for a PROCEEDINGS abstract or paper.
\acmConference[Conference acronym 'XX]{Conference title}{January 01--02,
  1900}{Someplace, NY}
\acmPrice{15.00}
\acmISBN{978-1-4503-XXXX-X/18/06}

%%
%% Submission ID.
%% Use this when submitting an article to a sponsored event. You'll
%% receive a unique submission ID from the organizers
%% of the event, and this ID should be used as the parameter to this command.
\acmSubmissionID{123-A56-BU3}

%%
%% The majority of ACM publications use numbered citations and
%% references.  The command \citestyle{authoryear} switches to the
%% "author year" style.
%%
%% If you are preparing content for an event
%% sponsored by ACM SIGGRAPH, you must use the "author year" style of
%% citations and references.
%% Uncommenting
%% the next command will enable that style.
\citestyle{acmauthoryear}


\begin{document}

\title{Coevolution of Camouflage}


%% Authors
\author{Craig Reynolds}
\email{cwr@red3d.com}
\orcid{0000-0001-8203-712X}
\affiliation{%
  \institution{unaffiliated researcher}
  \country{USA}
}

\renewcommand{\shortauthors}{Firstauthor et al.}

\begin{abstract}
  The abstract of your article
\end{abstract}

%%
%% Generate your CCSCML using http://dl.acm.org/ccs.cfm.
%%
\begin{CCSXML}
<ccs2012>
 <concept>
  <concept_id>10010520.10010553.10010562</concept_id>
  <concept_desc>Computer systems organization~Embedded systems</concept_desc>
  <concept_significance>500</concept_significance>
 </concept>
 <concept>
  <concept_id>10010520.10010575.10010755</concept_id>
  <concept_desc>Computer systems organization~Redundancy</concept_desc>
  <concept_significance>300</concept_significance>
 </concept>
 <concept>
  <concept_id>10010520.10010553.10010554</concept_id>
  <concept_desc>Computer systems organization~Robotics</concept_desc>
  <concept_significance>100</concept_significance>
 </concept>
 <concept>
  <concept_id>10003033.10003083.10003095</concept_id>
  <concept_desc>Networks~Network reliability</concept_desc>
  <concept_significance>100</concept_significance>
 </concept>
</ccs2012>
\end{CCSXML}

\ccsdesc[500]{Computer systems organization~Embedded systems}
\ccsdesc[300]{Computer systems organization~Redundancy}
\ccsdesc{Computer systems organization~Robotics}
\ccsdesc[100]{Networks~Network reliability}

%% Keywords
\keywords{relevant,key,words}

%% Teaser figure that appears on the top of the article.
%% Uncomment the includegraphics line to include an actual teaser image.
%% Make sure to fill out a description for accessibility
\begin{teaserfigure}
  %\includegraphics[width=\textwidth]{yourgraphic}
  \caption{Teaser figure}
  \Description{This is the teaser figure for the article.}
  \label{fig:teaser}
\end{teaserfigure}

\maketitle

%% The actual document with your content starts here

\section{Section}
Text

%% Acknowledgements
\begin{acks}
Acknowledgements
\end{acks}

%% Bibliography.
%% Uncomment the bibliography line and link to an actual bib file
\bibliographystyle{ACM-Reference-Format}
\bibliography{bibliography.bib}
% \bibliography{coc.bib}


%% Appendix
\appendix

\section{Additional Section}

Text

\end{document}
